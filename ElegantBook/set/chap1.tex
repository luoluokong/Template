\chapter{Group Theory}

\section{Group}
\begin{definition}[monoid]  \label{def: monoid}
    We say (G, $\dots$) is a monoid, if
    \begin{enumerate}
        \item $\forall x,y,z \in G, x(yz)=(xy)z$
        \item $\exists e \in G, ex=xe=e$
    \end{enumerate}
    where e is called an indentity element.
\end{definition}

Here, e is unique, since if e' is another identity element, then
\[
    e = ee{'} = e{'}
.\] 


\begin{definition}[group]  \label{def: group}
    We say (G, $\dots$) is a group, if
    \begin{enumerate}
        \item $\forall x,y,z \in G, x(yz)=(xy)z$
        \item $\exists e \in G, ex=xe=x$
        \item $\forall x \in G, \exists x^{-1}x=xx^{-1}=e$
    \end{enumerate}
    where $x^{-1}$ is called the inverse of x.\\
    We say G is a abelian, if moreover $\forall x,y \in G, xy=yx$.\\
    We say G is finite, if it is finite as a set. If so, we denote $\left|G\right|$ by the order of G, meaning the number of element of G.
\end{definition}

Here, $x^{-1}$ is unique for all x, since if y,z are inverses of x, then
\[
y=ye=y(xz)=(yx)z=ez=z
.\] 
Also, $(x^{-1})^{-1}=x$, since by definition,
\[
x^{-1}x=xx^{-1}=e
.\] 
And, $(xy)^{-1} = y^{-1}x^{-1}$, since




